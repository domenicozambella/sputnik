% !TEX root = sputnik.tex
\documentclass[sputnik.tex]{subfiles}
\begin{document}

\def\Fr{\mathop{\rm Fr}}

\def\vc{{\footnotesize VC}}
\def\nip{{\footnotesize NIP}}


\def\medrel#1{\parbox[t]{6ex}{$\displaystyle\hfil #1$}}
\def\ceq#1#2#3{\parbox[t]{15ex}{$\displaystyle #1$}\medrel{#2}{$\displaystyle #3$}}



\chapter{Packings and transversals}



%%%%%%%%%%%%%%%%%%%%%%%%%%%%%%%%%%%%%%%%%%%%%%%%%%%%%%
%%%%%%%%%%%%%%%%%%%%%%%%%%%%%%%%%%%%%%%%%%%%%%%%%%%%%%
\section{Linear programming / convex optimization}\label{LP}

The following is a classical result.
The proof may be found in any introductory text of linear programming.
There are many (sometimes non equivalent) ways to state it.
The following somewhat unusual phrasing is taken from Terence Tao's blog.
It stresses the analogies between Farkas' lemma and Hilbert's Nullstellensatz.
% (we have reversed strict and weak inequalities, but the proof is identical).

\begin{proposition}[(Farkas' Lemma)]
For $i\in[m]$ let $P_i:\RR^n\to\RR$ be affine linear functions.
Then the following are equivalent
\nobreak  
\begin{itemize}
\item[1.] $\displaystyle\bigwedge^m_{i=1}P_i(x)\ge 0$ has a solution $x\in\RR^n$;
\item[2.] there are no $0\le y\in\RR^m$ such that \smash{$\displaystyle\sum^m_{i=1}y_iP_i=-1$}.\QED
\end{itemize}
\end{proposition}



\begin{proposition}[(Farkas' Lemma, second formulation)]
Let $A$ be a $m\times n$ real matrix. 
%Let $\big\<A_{i,j}\ :\ i\in[m],\ j\in[n]\big\>$ be the the entries of a real matrix.
Let $b\in\RR^m$ be a column vector.
Then the following are equivalent
\nobreak  
\begin{itemize}
\item[1.] $Ax\ge b$ has a solution $x\in\RR^n$;
\item[2.] all $0\le y\in\RR^m$ such that $A^{\rm T}y=0$ also satisfy $b^{\rm T}y\ge0$.\QED
\end{itemize}
\end{proposition}


\begin{proposition}[(Duality for LP)]
Let $A$ be a $m\times n$ real matrix. Let $c\in\RR^m$ and $b\in\RR^n$ be column vectors. Then the following maximum and minimum exist and coincide.\nobreak  
% \begin{itemize}
% \item[1.] $\max\{\,c^{\rm T}\, x\ :\ \ A\,x\le b,\ 0\le x\}$;
% \item[2.] $\min\,\{\,b^{\rm T}\,y\ :\ A^{\rm T}y\ge c,\  0\le y\}$.\QED
% \end{itemize}
\begin{itemize}
\item[1.] $\min\,\{\,c^{\rm T}\, x\ :\ \ A\,x\ge b,\ 0\le x\}$;
\item[2.] $\max\{\,b^{\rm T}\,y\ :\ A^{\rm T}y\le c,\  0\le y\}$.\QED
\end{itemize}

\end{proposition}

% \begin{proof}
% For $i=1,\dots,m$ let 
% 
% \ceq{\hfill P_i\ :\ \RR^n}{\to}{\RR} 
% 
% \ceq{\hfill x\kern1.3ex}{\mapsto}{\Big(\sum^n_{j=1}A_{i,j}\,x_j\Big)-1}
% 
% Also for an given $y$ define
% 
% \ceq{\hfill P_{m+1}\ :\ \RR^n}{\to}{\RR} 
% 
% \ceq{\hfill x\kern1.3ex}{\mapsto}{\Big(\sum^n_{j=1}x_j\Big)-y}
% 
% the maximal $y$ such that the system of equation has a non negative solution. By Farkas's lemma this is the  maximal $y$ such that 
% \end{proof}

%%%%%%%%%%%%%%%%%%%%%%%
%%%%%%%%%%%%%%%%%%%%%%%
%%%%%%%%%%%%%%%%%%%%%%%
%%%%%%%%%%%%%%%%%%%%%%%
%%%%%%%%%%%%%%%%%%%%%%%

\section{Transversals and packings}\label{Transversals_Packings}

A subset $A\subseteq\U$ is a \emph{tansversal\/} if $\phi(A,b)\neq\0$ for every $b\in\V$. Equivalently, if $\phi(a,\V)_{a\in A}$ covers $\V$.
The \emph{transversal number\/} of $\phi$, denoted by $\tau(\phi)$, is the smallest cardinality of a transversal of $\phi$.

A subset $B\subseteq\V$ is a \emph{packing\/} if $\phi(\U,b)\cap\phi(\U,b')=\0$ for every $b,b'\in B$. Equivalently if $|\phi(a,B)|\le1$ for every $a\in\U$.
The \emph{packing number\/} of $\phi$, denoted by $\nu(\phi)$, is the largest cardinality of a packing  $B\subseteq\V$.

As any transversal is at least as large as any packing, we always have $\nu(\phi)\le\tau(\phi)$.
Very little can be said in the reverse direction in general.

A fractional multi-set $A$ over $\U$ is a \emph{fractional tansversal\/} if $|\phi(A, b)|\ge1$ for every $b\in\V$.
The \emph{fractional transversal number\/} of $\phi$, denoted by $\tau^*(\phi)$, is the  infimum of the size of the fractional transversals of $\phi$.

A fractional multi-set $B$ over $\V$ is a \emph{fractional packing\/} if $|\phi(a,B)|\le1$ for every $a\in\U$.
The \emph{fractional packing number\/} of $\phi$, denoted by $\nu^*(\phi)$, is the supremum of the size of the fractional packings of $\phi$.

\begin{example}
Let $\U=\RR^2$ and $\V$ is a set of $n$ lines in general position.
Let $\phi$ be the incidence relation.
%That is, any to lines intersect and every point is contained in at most to lines.
Then $\nu(\phi)=1$, as any two lines intersect.
And $\tau(\phi)=\lceil n/2\rceil$, as each point belongs to at most two lines.

Let $B$ assign $1/2$ to every line. Then $|\phi(a,B)|\le1$ holds because each point is contained in at most two lines.
Then $\nu^*(\phi)\ge n/2$.
It is easy to see that $\tau^*(\phi)= n/2$.
If $n$ is even, it is clear.
If $n$ is odd, take $3$ any lines, and assign $1/2$ to the three intersection points.
Proceed as in the even case with the other lines.\QED
\end{example}

\begin{exercise}
Let $\U=\RR$ and $\V$ is a set of finitely many closed intervals.
Let $\phi$ be the incidence relation.
Then $\nu(\phi)=\tau(\phi)$.
Hint: by induction on $\nu(\phi)$.
\end{exercise}

\begin{theorem}
For every finite incidence relation $\nu^*(\phi)=\tau^*(\phi)$.
\end{theorem}
\begin{proof}
Let $\U=\{a_1,\dots,a_m\}$ and $\V=\{b_1,\dots,b_n\}$.
Let $F$ be the $\{0,1\}$-valued incidence matrix $\phi(a_i,b_j)$.
A multi-set over $\U$ is a naturally associated to a vector $0\le x\in\RR^m$.
A multi-set over $\U$ is associated to a vector $0\le y\in\RR^n$. 
Then 


\ceq{\hfill\tau^*(\phi)}{=}{\min\,\big\{\,1_n^{\rm T}\;x\ :\ \ F\,x\ge 1_m,\ 0\le x\}.}

\ceq{\hfill\nu^*(\phi)}{=}{\max\big\{\,1_n^{\rm T}\ y\,\  :\ F^{\rm T}y\le 1_n,\,\  0\le y\};}



\end{proof}

\end{document}
