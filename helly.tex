% !TEX root = sputnik.tex
\documentclass[sputnik.tex]{subfiles}
\begin{document}

\def\Fr{\mathop{\rm Fr}}

\def\vc{{\footnotesize VC}}
\def\nip{{\footnotesize NIP}}


\def\medrel#1{\parbox[t]{6ex}{$\displaystyle\hfil #1$}}
\def\ceq#1#2#3{\parbox[t]{15ex}{$\displaystyle #1$}\medrel{#2}{$\displaystyle #3$}}



\chapter{Packings and transversals}



%%%%%%%%%%%%%%%%%%%%%%%%%%%%%%%%%%%%%%%%%%%%%%%%%%%%%%
%%%%%%%%%%%%%%%%%%%%%%%%%%%%%%%%%%%%%%%%%%%%%%%%%%%%%%
\section{Linear programming / convex optimization}\label{LP}

The following is a classical result.
The proof may be found in any introductory text of linear programming.
There are many (sometimes non equivalent) ways to state it.
The following somewhat unusual phrasing is taken from Terence Tao's blog.
It stresses the analogies between Farkas' lemma and Hilbert's Nullstellensatz.
% (we have reversed strict and weak inequalities, but the proof is identical).

\begin{proposition}[(Farkas' Lemma)]
For $i\in[m]$ let $P_i:\RR^n\to\RR$ be affine linear functions.
Then the following are equivalent
\nobreak  
\begin{itemize}
\item[1.] $\displaystyle\bigwedge^m_{i=1}P_i(x)\ge 0$ has a solution $x\in\RR^n$;
\item[2.] there are no $0\le y\in\RR^m$ such that \smash{$\displaystyle\sum^m_{i=1}y_iP_i=-1$}.\QED
\end{itemize}
\end{proposition}



\begin{proposition}[(Farkas' Lemma, second formulation)]
Let $A$ be a $m\times n$ real matrix. 
%Let $\big\<A_{i,j}\ :\ i\in[m],\ j\in[n]\big\>$ be the the entries of a real matrix.
Let $b\in\RR^m$ be a column vector.
Then the following are equivalent
\nobreak  
\begin{itemize}
\item[1.] $Ax\ge b$ has a solution $x\in\RR^n$;
\item[2.] all $0\le y\in\RR^m$ such that $A^{\rm T}y=0$ also satisfy $b^{\rm T}y\ge0$.\QED
\end{itemize}
\end{proposition}


\begin{proposition}[(Duality for LP)]
Let $A$ be a $m\times n$ real matrix. Let $c\in\RR^m$ and $b\in\RR^n$ be column vectors. Then the following maximum and minimum exist and coincide.\nobreak  
% \begin{itemize}
% \item[1.] $\max\{\,c^{\rm T}\, x\ :\ \ A\,x\le b,\ 0\le x\}$;
% \item[2.] $\min\,\{\,b^{\rm T}\,y\ :\ A^{\rm T}y\ge c,\  0\le y\}$.\QED
% \end{itemize}
\begin{itemize}
\item[1.] $\min\,\{\,c^{\rm T}\, x\ :\ \ A\,x\ge b,\ 0\le x\}$;
\item[2.] $\max\{\,b^{\rm T}\,y\ :\ A^{\rm T}y\le c,\  0\le y\}$.\QED
\end{itemize}

\end{proposition}

% \begin{proof}
% For $i=1,\dots,m$ let 
% 
% \ceq{\hfill P_i\ :\ \RR^n}{\to}{\RR} 
% 
% \ceq{\hfill x\kern1.3ex}{\mapsto}{\Big(\sum^n_{j=1}A_{i,j}\,x_j\Big)-1}
% 
% Also for an given $y$ define
% 
% \ceq{\hfill P_{m+1}\ :\ \RR^n}{\to}{\RR} 
% 
% \ceq{\hfill x\kern1.3ex}{\mapsto}{\Big(\sum^n_{j=1}x_j\Big)-y}
% 
% the maximal $y$ such that the system of equation has a non negative solution. By Farkas's lemma this is the  maximal $y$ such that 
% \end{proof}

%%%%%%%%%%%%%%%%%%%%%%%
%%%%%%%%%%%%%%%%%%%%%%%
%%%%%%%%%%%%%%%%%%%%%%%
%%%%%%%%%%%%%%%%%%%%%%%
%%%%%%%%%%%%%%%%%%%%%%%

\section{Transversals and packings}\label{Transversals_Packings}

A subset $A\subseteq\U$ is a \emph{tansversal\/} if $\phi(A,b)\neq\0$ for every $b\in\V$. Equivalently, if $\phi(a,\V)_{a\in A}$ covers $\V$.
The \emph{transversal number\/} of $\phi$, denoted by $\tau(\phi)$, is the smallest cardinality of a transversal of $\phi$.

A subset $B\subseteq\V$ is a \emph{packing\/} if $\phi(\U,b)\cap\phi(\U,b')=\0$ for every $b,b'\in B$. Equivalently if $|\phi(a,B)|\le1$ for every $a\in\U$.
The \emph{packing number\/} of $\phi$, denoted by $\nu(\phi)$, is the largest cardinality of a packing  $B\subseteq\V$.

As any transversal is at least as large as any packing, we always have $\nu(\phi)\le\tau(\phi)$.
Very little can be said in the reverse direction in general.

A fractional multi-set $A$ over $\U$ is a \emph{fractional tansversal\/} if $|\phi(A, b)|\ge1$ for every $b\in\V$.
The \emph{fractional transversal number\/} of $\phi$, denoted by $\tau^*(\phi)$, is the  infimum of the size of the fractional transversals of $\phi$.

A fractional multi-set $B$ over $\V$ is a \emph{fractional packing\/} if $|\phi(a,B)|\le1$ for every $a\in\U$.
The \emph{fractional packing number\/} of $\phi$, denoted by $\nu^*(\phi)$, is the supremum of the size of the fractional packings of $\phi$.

\begin{example}
Let $\U=\RR^2$ and $\V$ is a set of $n$ lines in general position.
Let $\phi$ be the incidence relation.
%That is, any to lines intersect and every point is contained in at most to lines.
Then $\nu(\phi)=1$, as any two lines intersect.
And $\tau(\phi)=\lceil n/2\rceil$, as each point belongs to at most two lines.

Let $B$ assign $1/2$ to every line. Then $|\phi(a,B)|\le1$ holds because each point is contained in at most two lines.
Then $\nu^*(\phi)\ge n/2$.
It is easy to see that $\tau^*(\phi)= n/2$.
If $n$ is even, it is clear.
If $n$ is odd, take $3$ any lines, and assign $1/2$ to the three intersection points.
Proceed as in the even case with the other lines.\QED
\end{example}

\begin{exercise}
Let $\U=\RR$ and $\V$ is a set of finitely many closed intervals.
Let $\phi$ be the incidence relation.
Then $\nu(\phi)=\tau(\phi)$.
Hint: use induction on $\nu(\phi)$.
\end{exercise}

\begin{theorem}
For every finite incidence relation $\nu^*(\phi)=\tau^*(\phi)$ and this value is rational.
\end{theorem}
\begin{proof}
Let $\U=\{a_1,\dots,a_m\}$ and $\V=\{b_1,\dots,b_n\}$.
Let $F$ be the $\{0,1\}$-valued incidence matrix $\phi(a_i,b_j)$.
A multi-set over $\U$ is a naturally associated to a vector $0\le x\in\RR^m$.
A multi-set over $\U$ is associated to a vector $0\le y\in\RR^n$.
Then it is easy to verify that

\ceq{\hfill\tau^*(\phi)}{=}{\min\,\big\{\,1_n^{\rm T}\;x\ :\ \ F\,x\ge 1_m,\ 0\le x\}.}

\ceq{\hfill\nu^*(\phi)}{=}{\max\big\{\,1_n^{\rm T}\ y\,\  :\ F^{\rm T}y\le 1_n,\,\  0\le y\};}

Therefore, by he duality of linear programming  $\nu^*(\phi)=\tau^*(\phi)$.

As $\tau^*(\phi)$ is the minumum of the linear function $x\mapsto 1_n^{\rm T}\;x$ over a polyhedron, such minumum is attained at  vertex.
The inequalities describing the polyhedron have rational coefficients, so also the vertices have rational coordinates.
\end{proof}

\section{$\epsilon$-Nets}

Fix a set system $\Phi$ and let $\Pr$ be a probability measure on $\U$ such that all sets in $\Phi$ are measurable.
Fix also $\epsilon>0$.
An \emph{$\epsilon$-net\/} is a set $A\subseteq\U$ that intersects all sets in $\Phi$ of measure at least $\epsilon$.
In other words, an $\epsilon$-net is a transversal for the set system $\Phi_\epsilon=\big\{\B\in\Phi\,:\,\Pr(\B)\ge\epsilon\big\}$.

\begin{proposition}
For any $\epsilon>0$ and $d\in\omega$ there is some $N=N(\epsilon,k)$ such that for every set system $\Phi$ of \vc-dimension $\le k$, there is an $\epsilon$-net of cardinality $\le N$.
\end{proposition}

\begin{proof}

\end{proof}


\begin{proposition}
Let $\Phi$ be a finite set system with \vc-dimension $\le k$. Then 

\ceq{\hfill\tau(\Phi)}{\le}{ c\,\tau^*(\Phi)\,\ln\tau^*(\Phi).}
\end{proposition}

\begin{proof}
As 
\end{proof}

\section{Helly-type properies}

We now investigate methods of bounding $\tau^*(\phi)=\nu^*(\phi)$.
Recall a classical theorem of Helly.

\begin{proposition}[(Helly Theorem)]
Let $\Phi$ be a finite family of convex sets in $\RR^d$.
Assume that any $d+1$ sets from $\Phi$ have non empty intersection.
Then the whole family $\Phi$ has non empty intersection.
\end{proposition}

Note that Helly's theorem does not hold for families of finite $\vc$-dimension.
A counter example of \vc-dimension $2$ can be constructed with a family containing sets that are unions of two finite intervals of the real line.

The following property is more robust.

\begin{definition}
Let $\Phi$ be an infinite family of sets.
We say that $\Phi$ \emph{has fractional Helly number $k$\/} if for every $\alpha>0$ there is a $\beta>0$ such that:

if $S_i\in\Phi$, for $i=1,\dots,n$ are such that

\ceq{\hfill \bigcap_{i\in I}S_i}{\neq}{\0}\hfill for at least \ $\displaystyle\alpha{n\choose k}$ \ sets \ $\displaystyle I\in{[n]\choose k}$

then

\ceq{\hfill\bigcap_{i\in J}S_i}{\neq}{\0}\hfill for some $J\subseteq [n]$ of cardinality $\ge\beta\,n$.

We say that $\Phi$ has the \emph{fractional Helly property\/} if it has some finite Helly number.
The \emph{fractional Helly number\/} of $\Phi$ is the smallest number $k$ above.
\QED
\end{definition}

\begin{theorem}[(Matou\v{s}ek~\cite{M2004})]
Let $\Phi$ be a set system with $\pi^*(\phi)\in o(n^k)$.
Then $\Phi$ has fractional Helly number $\le k$.
\end{theorem}

Recall that $\pi^*(\phi)\in o(n^k)$ means that $\lim_{n\to\infty}\pi^*(\phi)/n^k=0$.
Which occurs in particular when the \vc-density is $<k$ or when the \vc-dimension is $\le k$.
\begin{proof}
Let $\Phi$ and $k$ be as in the assumptions of the theorem.
Let $\alpha$ be arbitrary and set $\beta=1/2m$ where $m$ is such that 

\ceq{\hfill \pi^*(m)}{<}{\frac\alpha4{m\choose k}.}

Fix some  $S_i\in\Phi$, for $i=1,\dots,n$ and assume
 
\ceq{\hfill \bigcap_{i\in I}S_i}{\neq}{\0}\quad for at least \ $\displaystyle\alpha{n\choose k}$ \ sets \ $\displaystyle I\in{[n]\choose k}$

We will abbreviate the intersection above with $S_I$.
We need to show that there is a set $J\subseteq [n]$ of cardinality $\ge\beta\,n$ such that $S_J\neq0$.
So, assume not and reason for a contradiction.
Note that we can assume $n>2m$ because for $n\le2m$ we can take take $|J|=1$.

Identify $J\subseteq[n]$ with subsets of $\V$.
We will find a set $J\subseteq[n]$ of cardinality $m$ with many $\phi^*$-definable subsets, more than $\pi^*(m)$, a contradiction.

Let $P$ be the set of pairs $I\subseteq J\subseteq[n]$ such that $|I|=k$ and $|J|=m$.
We say that a pair $I\subseteq J$ in $P$ is \textit{good\/} if there is $a\in\U$ such that $I=\{i\in J\ :\ a\in S_i\}$.
That is, $I$ is a $\phi^*$-definable subset of $J$.

\textit{Claim.} Assume on $P$ the uniform probability.
Then the probability that a random pair is good is $\ge\alpha/4$.

Assume the claim and proceed with the proof.
There is a $J\in{[n]\choose m}$ with at least $\frac\alpha4{m\choose k}$ good subsets.
This yields the required contradiction.

We now prove the claim.
First we choose at random $I\in{[n]\choose k}$ of cardinality $k$, then we choose at random $m-k$ elements from $[n]\sm I$.
By assumption, the probability that $S_I\neq\0$ is at least $\alpha$. If  $S_I\neq\0$, we fix $a\in S_I$.
By assumption $a$ is contained in fewer than $\beta n$ sets $S_i$.
Then the probability that all of the sets $S_i\in J\sm I$ contains $a$ is at most

\ceq{\hfill\frac{\displaystyle{\beta n\choose m-k}}{\displaystyle{n-k\choose m-k}}}{\le}{\prod^{m-k-1}_{i=0}\frac{\beta n-i}{n-k-i}}
\medrel{\le}
$\displaystyle$







We bound from below the probability that a pair $I\subseteq J\subset[n]$ is good. 
\end{proof}


\section{The (p,q)-theorem}

For integers $p\ge q$ we say that $\phi$ has the \emph{$(p,q)$-property\/} if out of any $p$ definable sets there are $q$ sets with non empty intersection.

\begin{theorem}[(Alon, Kleitman + Matou\v sek)] Let $p\ge q\ge d+1$ be natural numbers.
Then there is anuber $N=N(d,p,q)$ such that if $\phi$ has \vc-codensity $\le d$ satisfies the $(p,q)$-property then $\tau(\phi)\le N$.
\end{theorem}

\begin{proof}
As we are not trying to optimize $N$, we may prove the theorem for $q=d+1$.
\end{proof}
\section{Notes and references}

\begin{biblist}[]\normalsize
\bib{M2004}{article}{
   author={Matou\v sek, Ji\v r\'\i },
   title={Bounded VC-dimension implies a fractional Helly theorem},
   journal={Discrete Comput. Geom.},
   volume={31},
   date={2004},
   number={2},
   pages={251--255},
%   issn={0179-5376},
%   doi={10.1007/s00454-003-2859-z
   }
\end{biblist}

\end{document}
