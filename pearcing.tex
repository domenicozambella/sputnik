% !TEX root = sputnik.tex
\documentclass[sputnik.tex]{subfiles}
\begin{document}

\def\vc{{\footnotesize VC}}
\def\nip{{\footnotesize NIP}}


\def\medrel#1{\parbox[t]{6ex}{$\displaystyle\hfil #1$}}
\def\ceq#1#2#3{\parbox[t]{25ex}{$\displaystyle #1$}\medrel{#2}{$\displaystyle #3$}}

\definecolor{brown}{RGB}{150, 50, 10}
\definecolor{green}{RGB}{10,120, 20}
\def\mr{\color{brown}}
\def\gr{\color{green}}


%%%%%%%%%%%%%%%%%%%%%%%%%%%
%%%%%%%%%%%%%%%%%%%%%%%%%%%
%%%%%%%%%%%%%%%%%%%%%%%%%%%
%%%%%%%%%%%%%%%%%%%%%%%%%%%
%%%%%%%%%%%%%%%%%%%%%%%%%%%
\chapter{The (p,q)-theorem}\label{qq}

After Pierre Simon, after Jiri Matousek, after Noga Alon and Daniel Kleitman.

Let $q\le p<\omega$ we say that $\Phi$ has the $(p,q)$-property if out of every $p$ sets in $\Phi$ some $q$ have non empty intersection.
It has the dual $(p,q)$-property if for every $p$ point in $\U$ some $q$ are belong to the same set in $\Phi$.


The following a is a particular case of a theorem of  Matousek (based on a proof of Noga Alon and Daniel Kleitman).
The proof by Pierre Simon is simpler.

\begin{theorem}\label{thm_qq}
Let $\U$ be finite and let $\Phi$ have \vc-dimension $k<\omega$.
There are some integers $q$ and $h$ that depends only on $k$ such that, if every $B\subseteq\U$ of cardinality $q$ is a subset of some $\phi\in\Phi$, then $\U$ is covered by some $X\subseteq\Phi$ of cardinality $h$.
\end{theorem}

This theorem is often stated in the dual form which explains why it is sometimes called a piercing (or Helly-type) theorem: there are some integers $q$ and $h$ that depends only on $k$ such that, if every $q$ sets in $\Phi$ have non-empty intersection, then there is a $B\subseteq\U$ of cardinality $h$ that intersects every $\phi\in\Phi$.
See Exercise~\ref{ex_dual_qqthm}.

% \begin{proof}
% Let $\<a_i:i<m\>$ and $\<\phi_j:j<n\>$ enumerate $\U$ and $\Phi$ without repetitions.
Recall that the dual system is $\U^*=\Phi$ and $\Phi\!^{*}\!=\big\{\Phi_i: i<m\big\}$, where $\Phi_i=\big\{\phi\in\Phi: a_i\in\phi\big\}$.
By Proposition~\ref{prop_vc*} the dual system has \vc-dimension $2^k$.
Choose some $\epsilon=1/2$.
From Theorem~\ref{thm_epsilon_approx} we obtain an $h$ such that every probability measure $\mu^*\!$ on \ $\Phi$ has an $\epsilon\jj$ap\-prox\-i\-ma\-tion of cardinality $h$.
As $\epsilon$ is fixed, this $h$ only depends on $k$.
Then there is a muti-set $X:\Phi\to\omega$ of cardinality $h$ such that
% 
% \parbox{10ex}{$\sharp$}
% $\displaystyle\left|\mu\!^*\!(\Phi_i)\ -\ \frac{|X\cap\Phi_i|}{h}\right|\medrel{\le}\epsilon.$
% 
% Recall how we read notation with multi-sets
% 
% \parbox{10ex}{~}
% $\displaystyle|X\cap\Phi_i|\medrel{=}\sum_{a_i\in\phi} X(\phi)$
% 
% Now, suppose that the probability measure on $\Phi$ is such that $\mu\!^*\!(\Phi_i)>\epsilon$ for all $i<m$, then the set $Y=\big\{\phi\in\Phi\ :\ X(\phi)\neq\0\big\}$ intersects every $\phi\in\Phi$.
Moreover $|Y|\le h$ as required.
% 
% For $i<m$ and $j<n$ let $P_{i,j}=1$ if $a_i\in\phi_j$ and $0$ otherwise.
Suppose we can find some $x_j\in\RR$ such that for every $i<m$
% 
% \parbox{10ex}{$\flat$}
% $\displaystyle\sum_{j<n}\big(P_{i,j}-\epsilon\big)x_j\medrel{>}0$
% 
% Then we can assume that all $x_j$ are positive.
In fact,
% 
% \parbox{10ex}{~}
% $\displaystyle\sum_{j<n}P_{i,j}\medrel{\ge}1$
% 
% (because every $a_i$ belongs to some $\phi_j$) so we can translate translate solutions of any positive quantity.
We can also rescale solutions so, setting $\mu\!^*(\phi_j)\deq x_j$, we obtain a well-defined probability measure
% 
% \parbox{10ex}{~}
% $\displaystyle\mu\!^*(\Phi_i)\medrel{=}\sum_{j<n}P_{i,j}\,x_j\medrel{>}\epsilon$
% 
% We only have to prove that equation $\flat$ has a solution.
It suffices to show that clause \ssf{2} of Farkas' Lemma cannot obtain, that is, for every $\lambda_i\in\RR_+$ there are some $x_j\in\RR_+$ such that
% 
% \parbox{10ex}{$\flat\flat$}
% $\displaystyle\sum_{i<m}\lambda_i\sum_{j<n}\big(P_{i,j}-\epsilon\big)x_j\medrel{>}0$.
% 
% As we can assume the $\lambda_i$ add to $1$, setting $\mu(a_i)\deq\lambda_i$, we obtain a probability measure on $\U$.
Note that for every $j$ we have
% 
% \parbox{10ex}{~}
% $\displaystyle\mu(\phi_j)\medrel{=}\sum_{i<m}\lambda_i P_{i,j}.$
% 
% By Theorem~\ref{thm_epsilon_approx}, once again, there is a $q$ such that for every probability measure $\mu$ on $\U$ there is a multi-set $B:\U\to\omega$ of cardinality $q$ such that
% 
% \parbox{10ex}{$\sharp\sharp$}
% $\displaystyle\left|\mu(\phi_j)\ -\ \frac{\big|B\cap\phi_j\big|}{q}\right|\medrel{\le}\epsilon.$
% 
% As $\epsilon$ is fixed, $q$ only depends on $k$.
By assumption, there is a $\check\jmath$ such that $\big\{a:B(a)\neq0\big\}\subseteq\phi_{\check\jmath}$ hence $|B\cap\phi_{\check\jmath}|=q$.
Therefore $\mu(\phi_{\check\jmath})>1-\epsilon$.
Let $\check x_j$ be $1$ for $j=\check\jmath$ and $0$ otherwise.
We claim that the tuple $\<x_j:j<n\>$ is a solution of $\flat\flat$.
In fact
% 
% \parbox{10ex}{~}
% $\displaystyle\sum_{i<m}\lambda_i\sum_{j<n}\big(P_{i,j}-\epsilon\big)\check x_j\medrel{=}{\sum_{i<m}\lambda_i\big(P_{i,\check\jmath}-\epsilon\big)}\medrel{=}\mu(\phi_{\check\jmath})-\epsilon\medrel{\ge}1-\epsilon\medrel{>}0$.
% 
% This concludes the proof.
% \end{proof}



\begin{proof}
Let $\<a_i:i<m\>$ and $\<\phi_j:j<n\>$ enumerate $\U$ and $\Phi$ without repetitions.
Recall that the dual system is $\U\!^*=\Phi$ and $\Phi\!^{*}\!=\big\{\Phi_i: i<m\big\}$, where $\Phi_i=\big\{\phi\in\Phi: a_i\in\phi\big\}$.
By Proposition~\ref{prop_vc*} the dual system has \vc-dimension $2^k$.
Choose some $\epsilon=1/2$.
From Corollary~\ref{coroll_epsilon_net} we obtain an $h$ such that every probability measure $\mu\!^*$ on \ $\Phi$ admits an $\epsilon\jj$net of cardinality $h$.
As $\epsilon$ is fixed, this $h$ only depends on $k$.
Restating this explicitly there is a set $X\subseteq\U\!^*$ of cardinality $h$ such that

\parbox{10ex}{$\sharp$}
$\displaystyle\left|\mu\!^*\!(\Phi_i)\right|>\epsilon\medrel{\IMP}X\cap\Phi_i\neq\0$

So, if the probability measure on $\Phi$ is such that $\mu\!^*\!(\Phi_i)>\epsilon$ for all $i<m$, then $X\subseteq\Phi$ cover $\U$ as required by the theorem.
So we only need to find such a probability measure.

For $i<m$ and $j<n$ let $P_{i,j}=1$ if $a_i\in\phi_j$ and $0$ otherwise.
Suppose we can find some $x_j\in\RR$ such that for every $i<m$

\parbox{10ex}{$\flat$}
$\displaystyle\sum_{j<n}\big(P_{i,j}-\epsilon\big)x_j\medrel{>}0$

Then we can assume that all $x_j$ are positive.
In fact,

\parbox{10ex}{~}
$\displaystyle\sum_{j<n}P_{i,j}\medrel{\ge}1$

(because every $a_i$ belongs to some $\phi_j$) so we can translate solutions of any positive quantity.
We can also rescale solutions so, setting $\mu\!^*(\phi_j)\deq x_j$, we obtain a well-defined probability measure

\parbox{10ex}{~}
$\displaystyle\mu\!^*(\Phi_i)\medrel{=}\sum_{j<n}P_{i,j}\,x_j\medrel{>}\epsilon$

We only have to prove that equation $\flat$ has a solution.
It suffices to show that clause \ssf{2} of Farkas' Lemma cannot obtain, that is, for every $\lambda_i\in\RR_+$ there are some $x_j\in\RR_+$ such that

\parbox{10ex}{$\flat\flat$}
$\displaystyle\sum_{i<m}\lambda_i\sum_{j<n}\big(P_{i,j}-\epsilon\big)x_j\medrel{>}0$.

As we can assume the $\lambda_i$ add to $1$, setting $\mu(a_i)\deq\lambda_i$, we obtain a probability measure on $\U$.
Note that for every $j$ we have

\parbox{10ex}{~}
$\displaystyle\mu(\phi_j)\medrel{=}\sum_{i<m}\lambda_i P_{i,j}.$

Apply Corollary~\ref{coroll_epsilon_net} once again to the set-system $\<\U,\neg\Phi\>$.
There is a $q$ such that for every probability measure $\mu$ on $\U$ there is a set $B\subseteq\U$ of cardinality $q$ such that

\parbox{10ex}{$\sharp\sharp$}
$\displaystyle\mu(\neg\phi_j)>\epsilon\medrel{\IMP}B\not\subseteq\phi_j$

As $\epsilon$ is fixed, $q$ only depends on $k$.
By assumption, there is a $\check\jmath$ such that $B\subseteq\phi_{\check\jmath}$ and therefore $\mu(\phi_{\check\jmath})\ge1-\epsilon$.
Let $\check x_j$ be $1$ for $j=\check\jmath$ and $0$ otherwise.
We claim that the tuple $\<x_j:j<n\>$ is a solution of $\flat\flat$.
In fact

\parbox{10ex}{~}
$\displaystyle\sum_{i<m}\lambda_i\sum_{j<n}\big(P_{i,j}-\epsilon\big)\check x_j\medrel{=}{\sum_{i<m}\lambda_i\big(P_{i,\check\jmath}-\epsilon\big)}\medrel{=}\mu(\phi_{\check\jmath})-\epsilon\medrel{\ge}1-\epsilon\medrel{>}0$.
This concludes the proof.
\end{proof}

% \begin{proof}
% \def\myC{\raisebox{-.5ex}{\Bigg[}}
% \def\myJ{\raisebox{-.5ex}{\Bigg]}}
% 
% By induction on $n$.
It will help induction 
% 
% \begin{itemize}
% \item[1.] $\displaystyle\bigvee_{j<m}\myC\bigwedge_{i<m}P_{i,j}(x)> 0\myJ$ for some $x\in\RR^n$;
% \item[2.] for every $j<m$ there are some $\lambda_i\in\RR_+$ such that \smash{$\displaystyle\sum_{i<m}\lambda_iP_{i,j}(x)\le 0$} for every $x\in\RR^n$.
% \end{itemize}
% 
% Let $|x|=n$ and $|y|=1$.
Up to rescaling we can assume that $P_{i,j}(x,y)$ have the form 
% 
% \parbox[t]{10ex}{\hfil$\displaystyle\bigvee_{j<m}$}
% $\displaystyle\myC\;\bigwedge_{i\in I_1}P_{i,j}(x)+y>0\medrel{\wedge}\bigwedge_{i\in I_2}y-P_{i,j}(x)>0\medrel{\wedge}\bigwedge_{i\in I_3}P_{i,j}(x)>0\myJ$
% 
% It is easy to see that these equations can ve solved if we can first solve in the following equations in $x$   
% 
% \parbox[t]{10ex}{\hfil$\displaystyle\bigvee_{j<m}$}
% $\displaystyle\min_{i\in I_1} P_{i,j}(x)\ \ge\ \max_{i\in I_2}P_{i,j}(x)\wedge\myC\,\bigwedge_{i\in I_3}P_{i,j}(x)>0\myJ$
% 
% 
% The former can be written 
% 
% \end{proof}

\begin{exercise}\label{ex_dual_qqthm}
Let $k$, $q$ and $h$ be some integers as in Theorem~\ref{thm_qq} and let $\Phi$ be a set-system with dual \vc-dimension $k$.
Suppose that any $q$ sets in $\Phi$ have non-empty intersection and prove that there is some set $B\subseteq\U$ of cardinality $h$ that intersects every $\phi\in\Phi$.\QED
\end{exercise}




\section{Appendix: Farkas' lemma}\label{appendix}

Below, an affine version of Farkas' Lemma tailored to our purposes.

\begin{proposition}
Fix some $v_1,\dots,v_n,u\in\QQ^k$ and let $r_1,\dots,r_n,s\in\QQ$.
Let 

\hfil $X(r_1,\dots,r_n)=\{x\in\QQ^k\ :\ r_i\le v_i\cdot x, \textrm{ for every }i=1,\dots,n\}$.


Then the following are equivalent
\begin{itemize}
\item[1.] $s\le u\cdot x$ for all $x\in X(r_1,\dots,r_n)$;
\item[2.] there exist $q_1,\dots,q_n\in\QQ^+$ such that \smash{$\displaystyle\sum^n_{i=1} q_iv_i=u$}.
\end{itemize}
\end{proposition}
\begin{proof}
Implication \ssf{2}$\IMP$\ssf{1} is immediate.
To prove \ssf{1}$\IMP$\ssf{2} assume \ssf{1}.
We claim that
\begin{itemize}
\item[3.] $0\le u\cdot x$ for all $x\in X(0,\dots,0)$.
\end{itemize}
Suppose not and let $x\in X(0,\dots,0)$ be such that $u\cdot x<0$.
Let $y\in X(r_1,\dots,r_n)$.
Then $y+ax\in X(r_1\dots,r_n)$ for every $a\in\QQ^+$.
From \ssf{1} we obtain $s\le u\cdot (y+ax)$ which, for $a$ is sufficiently large, is a contradiction.


From \ssf{3} it follows that if $x\cdot v_i=0$ for $i=1,\dots,n$ then $0= u\cdot x$.
We can assume without loss of generality that $v_1,\dots,v_n$ are linearly independent.
Then by linear algebra \smash{$\sum q_iv_i=u$} for some $q_i\in\QQ$.
Now fix $i$ and verify that $q_i$ is non-negative.
Let $x_i$ such that $x_i\cdot v_j=1$, if $i=j$, and $0$ otherwise.
Then $0\le x_i\cdot u=q_i$.
\end{proof}



 
 
\end{document}
